\documentclass[]{article}
\usepackage{lmodern}
\usepackage{amssymb,amsmath}
\usepackage{ifxetex,ifluatex}
\usepackage{fixltx2e} % provides \textsubscript
\ifnum 0\ifxetex 1\fi\ifluatex 1\fi=0 % if pdftex
  \usepackage[T1]{fontenc}
  \usepackage[utf8]{inputenc}
\else % if luatex or xelatex
  \ifxetex
    \usepackage{mathspec}
  \else
    \usepackage{fontspec}
  \fi
  \defaultfontfeatures{Ligatures=TeX,Scale=MatchLowercase}
\fi
% use upquote if available, for straight quotes in verbatim environments
\IfFileExists{upquote.sty}{\usepackage{upquote}}{}
% use microtype if available
\IfFileExists{microtype.sty}{%
\usepackage{microtype}
\UseMicrotypeSet[protrusion]{basicmath} % disable protrusion for tt fonts
}{}
\usepackage[margin=1in]{geometry}
\usepackage{hyperref}
\hypersetup{unicode=true,
            pdfborder={0 0 0},
            breaklinks=true}
\urlstyle{same}  % don't use monospace font for urls
\usepackage{color}
\usepackage{fancyvrb}
\newcommand{\VerbBar}{|}
\newcommand{\VERB}{\Verb[commandchars=\\\{\}]}
\DefineVerbatimEnvironment{Highlighting}{Verbatim}{commandchars=\\\{\}}
% Add ',fontsize=\small' for more characters per line
\usepackage{framed}
\definecolor{shadecolor}{RGB}{248,248,248}
\newenvironment{Shaded}{\begin{snugshade}}{\end{snugshade}}
\newcommand{\KeywordTok}[1]{\textcolor[rgb]{0.13,0.29,0.53}{\textbf{#1}}}
\newcommand{\DataTypeTok}[1]{\textcolor[rgb]{0.13,0.29,0.53}{#1}}
\newcommand{\DecValTok}[1]{\textcolor[rgb]{0.00,0.00,0.81}{#1}}
\newcommand{\BaseNTok}[1]{\textcolor[rgb]{0.00,0.00,0.81}{#1}}
\newcommand{\FloatTok}[1]{\textcolor[rgb]{0.00,0.00,0.81}{#1}}
\newcommand{\ConstantTok}[1]{\textcolor[rgb]{0.00,0.00,0.00}{#1}}
\newcommand{\CharTok}[1]{\textcolor[rgb]{0.31,0.60,0.02}{#1}}
\newcommand{\SpecialCharTok}[1]{\textcolor[rgb]{0.00,0.00,0.00}{#1}}
\newcommand{\StringTok}[1]{\textcolor[rgb]{0.31,0.60,0.02}{#1}}
\newcommand{\VerbatimStringTok}[1]{\textcolor[rgb]{0.31,0.60,0.02}{#1}}
\newcommand{\SpecialStringTok}[1]{\textcolor[rgb]{0.31,0.60,0.02}{#1}}
\newcommand{\ImportTok}[1]{#1}
\newcommand{\CommentTok}[1]{\textcolor[rgb]{0.56,0.35,0.01}{\textit{#1}}}
\newcommand{\DocumentationTok}[1]{\textcolor[rgb]{0.56,0.35,0.01}{\textbf{\textit{#1}}}}
\newcommand{\AnnotationTok}[1]{\textcolor[rgb]{0.56,0.35,0.01}{\textbf{\textit{#1}}}}
\newcommand{\CommentVarTok}[1]{\textcolor[rgb]{0.56,0.35,0.01}{\textbf{\textit{#1}}}}
\newcommand{\OtherTok}[1]{\textcolor[rgb]{0.56,0.35,0.01}{#1}}
\newcommand{\FunctionTok}[1]{\textcolor[rgb]{0.00,0.00,0.00}{#1}}
\newcommand{\VariableTok}[1]{\textcolor[rgb]{0.00,0.00,0.00}{#1}}
\newcommand{\ControlFlowTok}[1]{\textcolor[rgb]{0.13,0.29,0.53}{\textbf{#1}}}
\newcommand{\OperatorTok}[1]{\textcolor[rgb]{0.81,0.36,0.00}{\textbf{#1}}}
\newcommand{\BuiltInTok}[1]{#1}
\newcommand{\ExtensionTok}[1]{#1}
\newcommand{\PreprocessorTok}[1]{\textcolor[rgb]{0.56,0.35,0.01}{\textit{#1}}}
\newcommand{\AttributeTok}[1]{\textcolor[rgb]{0.77,0.63,0.00}{#1}}
\newcommand{\RegionMarkerTok}[1]{#1}
\newcommand{\InformationTok}[1]{\textcolor[rgb]{0.56,0.35,0.01}{\textbf{\textit{#1}}}}
\newcommand{\WarningTok}[1]{\textcolor[rgb]{0.56,0.35,0.01}{\textbf{\textit{#1}}}}
\newcommand{\AlertTok}[1]{\textcolor[rgb]{0.94,0.16,0.16}{#1}}
\newcommand{\ErrorTok}[1]{\textcolor[rgb]{0.64,0.00,0.00}{\textbf{#1}}}
\newcommand{\NormalTok}[1]{#1}
\usepackage{graphicx,grffile}
\makeatletter
\def\maxwidth{\ifdim\Gin@nat@width>\linewidth\linewidth\else\Gin@nat@width\fi}
\def\maxheight{\ifdim\Gin@nat@height>\textheight\textheight\else\Gin@nat@height\fi}
\makeatother
% Scale images if necessary, so that they will not overflow the page
% margins by default, and it is still possible to overwrite the defaults
% using explicit options in \includegraphics[width, height, ...]{}
\setkeys{Gin}{width=\maxwidth,height=\maxheight,keepaspectratio}
\IfFileExists{parskip.sty}{%
\usepackage{parskip}
}{% else
\setlength{\parindent}{0pt}
\setlength{\parskip}{6pt plus 2pt minus 1pt}
}
\setlength{\emergencystretch}{3em}  % prevent overfull lines
\providecommand{\tightlist}{%
  \setlength{\itemsep}{0pt}\setlength{\parskip}{0pt}}
\setcounter{secnumdepth}{0}
% Redefines (sub)paragraphs to behave more like sections
\ifx\paragraph\undefined\else
\let\oldparagraph\paragraph
\renewcommand{\paragraph}[1]{\oldparagraph{#1}\mbox{}}
\fi
\ifx\subparagraph\undefined\else
\let\oldsubparagraph\subparagraph
\renewcommand{\subparagraph}[1]{\oldsubparagraph{#1}\mbox{}}
\fi

%%% Use protect on footnotes to avoid problems with footnotes in titles
\let\rmarkdownfootnote\footnote%
\def\footnote{\protect\rmarkdownfootnote}

%%% Change title format to be more compact
\usepackage{titling}

% Create subtitle command for use in maketitle
\newcommand{\subtitle}[1]{
  \posttitle{
    \begin{center}\large#1\end{center}
    }
}

\setlength{\droptitle}{-2em}

  \title{}
    \pretitle{\vspace{\droptitle}}
  \posttitle{}
    \author{}
    \preauthor{}\postauthor{}
    \date{}
    \predate{}\postdate{}
  

\begin{document}

\subsection{Carregando as bibliotecas necessárias para a análise do
modelo}\label{carregando-as-bibliotecas-necessarias-para-a-analise-do-modelo}

\begin{Shaded}
\begin{Highlighting}[]
\KeywordTok{library}\NormalTok{(ggplot2)}
\CommentTok{#install.packages("dplyr")}
\KeywordTok{library}\NormalTok{(dplyr)}
\end{Highlighting}
\end{Shaded}

\begin{verbatim}
## 
## Attaching package: 'dplyr'
\end{verbatim}

\begin{verbatim}
## The following objects are masked from 'package:stats':
## 
##     filter, lag
\end{verbatim}

\begin{verbatim}
## The following objects are masked from 'package:base':
## 
##     intersect, setdiff, setequal, union
\end{verbatim}

\begin{Shaded}
\begin{Highlighting}[]
\CommentTok{#install.packages("glmnet")}
\KeywordTok{library}\NormalTok{(glmnet)}
\end{Highlighting}
\end{Shaded}

\begin{verbatim}
## Loading required package: Matrix
\end{verbatim}

\begin{verbatim}
## Loading required package: foreach
\end{verbatim}

\begin{verbatim}
## Loaded glmnet 2.0-16
\end{verbatim}

\begin{Shaded}
\begin{Highlighting}[]
\CommentTok{#install.packages("ISLR")}
\KeywordTok{library}\NormalTok{(ISLR)}
\CommentTok{#install.packages("caTools")}
\KeywordTok{library}\NormalTok{(caTools)}
\CommentTok{#install.packages("randomForest")}
\KeywordTok{library}\NormalTok{(randomForest)}
\end{Highlighting}
\end{Shaded}

\begin{verbatim}
## randomForest 4.6-14
\end{verbatim}

\begin{verbatim}
## Type rfNews() to see new features/changes/bug fixes.
\end{verbatim}

\begin{verbatim}
## 
## Attaching package: 'randomForest'
\end{verbatim}

\begin{verbatim}
## The following object is masked from 'package:dplyr':
## 
##     combine
\end{verbatim}

\begin{verbatim}
## The following object is masked from 'package:ggplot2':
## 
##     margin
\end{verbatim}

\begin{Shaded}
\begin{Highlighting}[]
\CommentTok{#install.packages("caret")}
\KeywordTok{library}\NormalTok{(caret)}
\end{Highlighting}
\end{Shaded}

\begin{verbatim}
## Loading required package: lattice
\end{verbatim}

\subsection{Carregando e tratando a base que será
analisada}\label{carregando-e-tratando-a-base-que-sera-analisada}

\subsubsection{\texorpdfstring{Fonte dos dados +
\url{https://github.com/alsombra/MBA-Machine_Learning}}{Fonte dos dados + https://github.com/alsombra/MBA-Machine\_Learning}}\label{fonte-dos-dados-httpsgithub.comalsombramba-machine_learning}

\begin{Shaded}
\begin{Highlighting}[]
\KeywordTok{setwd}\NormalTok{(}\StringTok{"C:/Users/ameri/source/repos/github/americofreitasjr/mba-fgv-trabalho-analisepreditiva"}\NormalTok{)}
\NormalTok{df =}\StringTok{ }\KeywordTok{read.csv}\NormalTok{(}\StringTok{"data_tratada.csv"}\NormalTok{)}

\CommentTok{#Retirando a coluna X - primeira coluna, que é apenas um índice}
\NormalTok{df <-}\StringTok{ }\NormalTok{df[,}\OperatorTok{-}\DecValTok{1}\NormalTok{] }
\end{Highlighting}
\end{Shaded}

\subsection{Análise Descritiva da variável resposta
JobSatisfaction}\label{analise-descritiva-da-variavel-resposta-jobsatisfaction}

\subsubsection{A satisfação não está distribuida normalmente, mostrando
uma concentração de satisfação entre 6 e
8}\label{a-satisfacao-nao-esta-distribuida-normalmente-mostrando-uma-concentracao-de-satisfacao-entre-6-e-8}

\begin{Shaded}
\begin{Highlighting}[]
\KeywordTok{hist}\NormalTok{(df}\OperatorTok{$}\NormalTok{JobSatisfaction, }\DataTypeTok{main =} \StringTok{"Distribuição - JobSatisfaction"}\NormalTok{)}
\end{Highlighting}
\end{Shaded}

\includegraphics{Trabalho_files/figure-latex/unnamed-chunk-3-1.pdf}

\subsubsection{O boxplot mostra um outlier, que neste caso não
representará um problema para o
modelo}\label{o-boxplot-mostra-um-outlier-que-neste-caso-nao-representara-um-problema-para-o-modelo}

\begin{Shaded}
\begin{Highlighting}[]
\KeywordTok{boxplot}\NormalTok{(df}\OperatorTok{$}\NormalTok{JobSatisfaction)}
\end{Highlighting}
\end{Shaded}

\includegraphics{Trabalho_files/figure-latex/unnamed-chunk-4-1.pdf}

\subsection{Classificação Floresta
Aleatória}\label{classificacao-floresta-aleatoria}

\subsubsection{Nesta etapa estamos separando de maneira aleatória os
dados de treio e teste, e posteriormente aplicando o randonForest para
selecionar as variáveis mais releventes. Durante o processo
identificamos que o randonForest não aceita mais de 53 níveis de
classificação, porém aceitamos que a acurácia do resultado não seria
relevantemente afetada com apenas os 53 níveis, e prosseguimos com o
modelo de
treino.}\label{nesta-etapa-estamos-separando-de-maneira-aleatoria-os-dados-de-treio-e-teste-e-posteriormente-aplicando-o-randonforest-para-selecionar-as-variaveis-mais-releventes.-durante-o-processo-identificamos-que-o-randonforest-nao-aceita-mais-de-53-niveis-de-classificacao-porem-aceitamos-que-a-acuracia-do-resultado-nao-seria-relevantemente-afetada-com-apenas-os-53-niveis-e-prosseguimos-com-o-modelo-de-treino.}

\begin{Shaded}
\begin{Highlighting}[]
\CommentTok{# Dividindo o conjunto de dados no conjunto de treinamento e no conjunto de teste}
\KeywordTok{set.seed}\NormalTok{(}\DecValTok{123}\NormalTok{)}
\NormalTok{ind <-}\StringTok{ }\KeywordTok{sample}\NormalTok{(}\DecValTok{1}\OperatorTok{:}\KeywordTok{nrow}\NormalTok{(df), }\DataTypeTok{size =}\NormalTok{ .}\DecValTok{75}\OperatorTok{*}\KeywordTok{nrow}\NormalTok{(df), }\DataTypeTok{replace =}\NormalTok{ F)}
\NormalTok{training_set =}\StringTok{ }\NormalTok{df[ind,]}
\NormalTok{test_set =}\StringTok{ }\NormalTok{df[}\OperatorTok{-}\NormalTok{ind,]}

\CommentTok{# Identificando as variáveis com mais de 53 níveis (que a função randomForest não aceita) }
\NormalTok{ind_col <-}\StringTok{ }\OtherTok{NULL}
\ControlFlowTok{for}\NormalTok{ (i }\ControlFlowTok{in}\NormalTok{ (}\DecValTok{1}\OperatorTok{:}\KeywordTok{ncol}\NormalTok{(training_set))) \{}
  \KeywordTok{ifelse}\NormalTok{(}\KeywordTok{length}\NormalTok{(}\KeywordTok{levels}\NormalTok{(training_set[,i]))}\OperatorTok{<}\DecValTok{54}\NormalTok{,ind_col[i]<-}\OtherTok{TRUE}\NormalTok{,ind_col[i]<-}\OtherTok{FALSE}\NormalTok{)   }
\NormalTok{\}}

\CommentTok{# base de treino só com variáveis preditoras}
\NormalTok{training_set_pred <-}\StringTok{ }\NormalTok{training_set[,}\OperatorTok{-}\DecValTok{14}\NormalTok{]}

\CommentTok{# Fazendo a floresta aleatória só com as variáveis preditoras que tenham menos de 53 níveis}
\KeywordTok{set.seed}\NormalTok{(}\DecValTok{123}\NormalTok{)}
\NormalTok{classifier =}\StringTok{ }\KeywordTok{randomForest}\NormalTok{(}\DataTypeTok{x =}\NormalTok{ training_set_pred[,}\KeywordTok{which}\NormalTok{(ind_col[}\OperatorTok{-}\DecValTok{14}\NormalTok{]}\OperatorTok{==}\OtherTok{TRUE}\NormalTok{)],}
                          \DataTypeTok{y =}\NormalTok{ training_set}\OperatorTok{$}\NormalTok{JobSatisfaction,}
                          \DataTypeTok{ntree =} \DecValTok{100}\NormalTok{)}

\CommentTok{# Previsão dos resultados de teste}
\NormalTok{y_pred =}\StringTok{ }\KeywordTok{predict}\NormalTok{(classifier, }\DataTypeTok{newdata =}\NormalTok{ test_set)}

\CommentTok{# Fazendo a matriz de confusão}
\NormalTok{cm =}\StringTok{ }\KeywordTok{table}\NormalTok{(test_set[,}\DecValTok{14}\NormalTok{], y_pred)}
\NormalTok{cm_dat =}\StringTok{ }\KeywordTok{data.frame}\NormalTok{(cm)}
\KeywordTok{plot}\NormalTok{(test_set[,}\DecValTok{14}\NormalTok{], y_pred)}
\end{Highlighting}
\end{Shaded}

\includegraphics{Trabalho_files/figure-latex/unnamed-chunk-5-1.pdf}

\subsubsection{Escolhendo o número de
árvores}\label{escolhendo-o-numero-de-arvores}

\subsection{Ao avaliar o númrero de árvores, identificamos que a partir
dos 40 níveis o modelo não tem uma melhora na
performance}\label{ao-avaliar-o-numrero-de-arvores-identificamos-que-a-partir-dos-40-niveis-o-modelo-nao-tem-uma-melhora-na-performance}

\begin{Shaded}
\begin{Highlighting}[]
\KeywordTok{plot}\NormalTok{(classifier)}
\end{Highlighting}
\end{Shaded}

\includegraphics{Trabalho_files/figure-latex/unnamed-chunk-6-1.pdf}

\subsection{Escolhendo as variáveis mais significativas, tendo como
critério diminuição média na impureza dos nós (estatística de
Gini)}\label{escolhendo-as-variaveis-mais-significativas-tendo-como-criterio-diminuicao-media-na-impureza-dos-nos-estatistica-de-gini}

\begin{Shaded}
\begin{Highlighting}[]
\KeywordTok{varImpPlot}\NormalTok{(classifier)}
\end{Highlighting}
\end{Shaded}

\includegraphics{Trabalho_files/figure-latex/unnamed-chunk-7-1.pdf}

\begin{Shaded}
\begin{Highlighting}[]
\CommentTok{#varImp}
\CommentTok{# Tabela com o valor da estatística de Gini em cada variável}
\NormalTok{importance_dat =}\StringTok{ }\KeywordTok{data.frame}\NormalTok{(}\KeywordTok{importance}\NormalTok{(classifier))}
\KeywordTok{View}\NormalTok{(importance_dat)}
\end{Highlighting}
\end{Shaded}

\subsection{Modelo Linear}\label{modelo-linear}

\begin{Shaded}
\begin{Highlighting}[]
\CommentTok{#Selecionadas (a princípio) as variáveis com índice de Gini > 60}
\NormalTok{mod =}\StringTok{ }\KeywordTok{lm}\NormalTok{(JobSatisfaction }\OperatorTok{~}\StringTok{ }\NormalTok{CareerSatisfaction }\OperatorTok{+}\StringTok{ }\NormalTok{JobSeekingStatus }\OperatorTok{+}\StringTok{ }\NormalTok{YearsProgram}
         \OperatorTok{+}\StringTok{ }\NormalTok{YearsCodedJob }\OperatorTok{+}\StringTok{ }\NormalTok{HoursPerWeek }\OperatorTok{+}\StringTok{ }\NormalTok{MajorUndergrad}
         \CommentTok{#+ Currency }
         \OperatorTok{+}\StringTok{ }\NormalTok{CompanySize}
         \CommentTok{#+ WorkStart}
         \OperatorTok{+}\StringTok{ }\NormalTok{ResumePrompted }\OperatorTok{+}\StringTok{ }\NormalTok{CompanyType }\OperatorTok{+}\StringTok{ }\NormalTok{HighestEducationParents}
         \OperatorTok{+}\StringTok{ }\NormalTok{InfluenceWorkstation }\OperatorTok{+}\StringTok{ }\NormalTok{Overpaid }\OperatorTok{+}\StringTok{ }\NormalTok{HomeRemote,}
         \DataTypeTok{data=}\NormalTok{training_set)}
\CommentTok{#print(mod)}
\CommentTok{#summary(mod)}
\NormalTok{yhat <-}\StringTok{ }\KeywordTok{predict}\NormalTok{(mod, test_set)}
\KeywordTok{mean}\NormalTok{((yhat}\OperatorTok{-}\NormalTok{test_set}\OperatorTok{$}\NormalTok{JobSatisfaction)}\OperatorTok{^}\DecValTok{2}\NormalTok{)}
\end{Highlighting}
\end{Shaded}

\begin{verbatim}
## [1] 2.686845
\end{verbatim}

\subsubsection{Job Satisfaction - Base de
treino}\label{job-satisfaction---base-de-treino}

\begin{Shaded}
\begin{Highlighting}[]
\KeywordTok{par}\NormalTok{(}\DataTypeTok{mfrow=}\KeywordTok{c}\NormalTok{(}\DecValTok{2}\NormalTok{,}\DecValTok{2}\NormalTok{))}
\KeywordTok{hist}\NormalTok{(training_set}\OperatorTok{$}\NormalTok{JobSatisfaction, }\DataTypeTok{main=} \StringTok{"Job Satisfaction - Base de treino"}\NormalTok{, }\DataTypeTok{col=}\StringTok{"lightblue"}\NormalTok{)}
\end{Highlighting}
\end{Shaded}

\includegraphics{Trabalho_files/figure-latex/unnamed-chunk-9-1.pdf}

\subsubsection{Job Satisfaction - Base de
teste}\label{job-satisfaction---base-de-teste}

\begin{Shaded}
\begin{Highlighting}[]
\KeywordTok{hist}\NormalTok{(test_set}\OperatorTok{$}\NormalTok{JobSatisfaction, }\DataTypeTok{main=} \StringTok{"Job Satisfaction - Base de teste"}\NormalTok{, }\DataTypeTok{col=}\StringTok{"lightblue"}\NormalTok{)}
\end{Highlighting}
\end{Shaded}

\includegraphics{Trabalho_files/figure-latex/unnamed-chunk-10-1.pdf}

\subsubsection{Job Satisfaction - Previsão na base de
teste}\label{job-satisfaction---previsao-na-base-de-teste}

\begin{Shaded}
\begin{Highlighting}[]
\KeywordTok{hist}\NormalTok{(yhat, }\DataTypeTok{main=} \StringTok{"Job Satisfaction - Previsão na base de teste"}\NormalTok{, }\DataTypeTok{col=}\StringTok{"lightblue"}\NormalTok{)}
\end{Highlighting}
\end{Shaded}

\includegraphics{Trabalho_files/figure-latex/unnamed-chunk-11-1.pdf}
\#\#\# Job Satisfaction - Base de teste vs Previsão na base de teste

\begin{Shaded}
\begin{Highlighting}[]
\KeywordTok{plot}\NormalTok{(test_set}\OperatorTok{$}\NormalTok{JobSatisfaction, yhat, }\DataTypeTok{main=} \StringTok{"Job Satisfaction - Base de teste vs Previsão na base de teste"}\NormalTok{)}
\end{Highlighting}
\end{Shaded}

\includegraphics{Trabalho_files/figure-latex/unnamed-chunk-12-1.pdf}

\subsubsection{Métricas de desempenho do
modelo}\label{metricas-de-desempenho-do-modelo}

\paragraph{O erro médio residual e o
R2}\label{o-erro-medio-residual-e-o-r2}

\begin{Shaded}
\begin{Highlighting}[]
\CommentTok{#Erro médio residual }
\NormalTok{RMSE =}\StringTok{ }\ControlFlowTok{function}\NormalTok{(m, o)\{}
  \KeywordTok{sqrt}\NormalTok{(}\KeywordTok{mean}\NormalTok{((m }\OperatorTok{-}\StringTok{ }\NormalTok{o)}\OperatorTok{^}\DecValTok{2}\NormalTok{))}
\NormalTok{\}}
\CommentTok{#R2}
\NormalTok{R2 <-}\StringTok{ }\ControlFlowTok{function}\NormalTok{ (x, y) }\KeywordTok{cor}\NormalTok{(x, y) }\OperatorTok{^}\StringTok{ }\DecValTok{2}

\KeywordTok{data.frame}\NormalTok{(}
  \DataTypeTok{RMSE =} \KeywordTok{RMSE}\NormalTok{(yhat, test_set}\OperatorTok{$}\NormalTok{JobSatisfaction),}
  \DataTypeTok{Rsquare =} \KeywordTok{R2}\NormalTok{(yhat, test_set}\OperatorTok{$}\NormalTok{JobSatisfaction)}
\NormalTok{)}
\end{Highlighting}
\end{Shaded}

\begin{verbatim}
##      RMSE   Rsquare
## 1 1.63916 0.3701806
\end{verbatim}

\subsection{Modelo LASSO (glmnet com alpha
=1)}\label{modelo-lasso-glmnet-com-alpha-1}

\begin{Shaded}
\begin{Highlighting}[]
\CommentTok{#Erro médio residual }
\CommentTok{# Definindo base de teste e treino no formato pedido pelo glmnet}
\NormalTok{x <-}\StringTok{ }\KeywordTok{model.matrix}\NormalTok{(JobSatisfaction }\OperatorTok{~}\StringTok{ }\NormalTok{.,df)[,}\OperatorTok{-}\DecValTok{1}\NormalTok{]}
\NormalTok{x.treino <-}\StringTok{ }\NormalTok{x[ind,]}
\NormalTok{y.treino <-}\StringTok{ }\NormalTok{df}\OperatorTok{$}\NormalTok{JobSatisfaction[ind]}
\NormalTok{x.teste <-}\StringTok{ }\NormalTok{x[}\OperatorTok{-}\NormalTok{ind,]}
\NormalTok{y.teste <-}\StringTok{ }\NormalTok{df}\OperatorTok{$}\NormalTok{JobSatisfaction[}\OperatorTok{-}\NormalTok{ind]}

\CommentTok{# Escolha do grau de regularização (lambda)}
\NormalTok{cv.out <-}\StringTok{ }\KeywordTok{cv.glmnet}\NormalTok{(x.treino, y.treino, }\DataTypeTok{alpha=}\DecValTok{1}\NormalTok{)}
\KeywordTok{plot}\NormalTok{(cv.out)}
\end{Highlighting}
\end{Shaded}

\includegraphics{Trabalho_files/figure-latex/unnamed-chunk-14-1.pdf}

\begin{Shaded}
\begin{Highlighting}[]
\CommentTok{# Modelo}
\NormalTok{lasso.mod=}\KeywordTok{glmnet}\NormalTok{(}\DataTypeTok{x=}\NormalTok{x.treino, }\DataTypeTok{y=}\NormalTok{y.treino, }\DataTypeTok{alpha=}\DecValTok{1}\NormalTok{, }\DataTypeTok{lambda=}\NormalTok{cv.out}\OperatorTok{$}\NormalTok{lambda.min)}

\CommentTok{# vamos olhar os coeficientes deste modelo e erro no conjunto de validação}
\NormalTok{resultPredict <-}\StringTok{ }\KeywordTok{predict}\NormalTok{(lasso.mod, }\DataTypeTok{s=}\NormalTok{ cv.out}\OperatorTok{$}\NormalTok{lambda.min , }\DataTypeTok{type=}\StringTok{"coefficients"}\NormalTok{)}
\CommentTok{#resultPredict}
\KeywordTok{head}\NormalTok{(resultPredict)}
\end{Highlighting}
\end{Shaded}

\begin{verbatim}
## 6 x 1 sparse Matrix of class "dgCMatrix"
##                                                              1
## (Intercept)                                           2.896908
## ProgramHobbyYes, both                                 .       
## ProgramHobbyYes, I contribute to open source projects .       
## ProgramHobbyYes, I program as a hobby                 .       
## CountryArgentina                                      .       
## CountryArmenia                                        .
\end{verbatim}

\begin{Shaded}
\begin{Highlighting}[]
\NormalTok{yhat <-}\StringTok{ }\KeywordTok{predict}\NormalTok{(lasso.mod, }\DataTypeTok{s =}\NormalTok{ cv.out}\OperatorTok{$}\NormalTok{lambda.min, }\DataTypeTok{type=}\StringTok{"response"}\NormalTok{, }\DataTypeTok{newx=}\NormalTok{x.teste)}
\KeywordTok{mean}\NormalTok{((yhat}\OperatorTok{-}\NormalTok{y.teste)}\OperatorTok{^}\DecValTok{2}\NormalTok{)}
\end{Highlighting}
\end{Shaded}

\begin{verbatim}
## [1] 2.522256
\end{verbatim}

\begin{Shaded}
\begin{Highlighting}[]
\KeywordTok{par}\NormalTok{(}\DataTypeTok{mfrow=}\KeywordTok{c}\NormalTok{(}\DecValTok{2}\NormalTok{,}\DecValTok{2}\NormalTok{))}
\KeywordTok{hist}\NormalTok{(y.treino, }\DataTypeTok{main=} \StringTok{"Job Satisfaction - Base de treino"}\NormalTok{, }\DataTypeTok{col=}\StringTok{"lightblue"}\NormalTok{)}
\end{Highlighting}
\end{Shaded}

\includegraphics{Trabalho_files/figure-latex/unnamed-chunk-17-1.pdf}

\begin{Shaded}
\begin{Highlighting}[]
\KeywordTok{hist}\NormalTok{(y.teste, }\DataTypeTok{main=} \StringTok{"Job Satisfaction - Base de teste"}\NormalTok{, }\DataTypeTok{col=}\StringTok{"lightblue"}\NormalTok{)}
\end{Highlighting}
\end{Shaded}

\includegraphics{Trabalho_files/figure-latex/unnamed-chunk-18-1.pdf}

\begin{Shaded}
\begin{Highlighting}[]
\KeywordTok{hist}\NormalTok{(yhat, }\DataTypeTok{main=} \StringTok{"Job Satisfaction - Previsão na base de teste"}\NormalTok{, }\DataTypeTok{col=}\StringTok{"lightblue"}\NormalTok{)}
\end{Highlighting}
\end{Shaded}

\includegraphics{Trabalho_files/figure-latex/unnamed-chunk-19-1.pdf}

\begin{Shaded}
\begin{Highlighting}[]
\KeywordTok{plot}\NormalTok{(y.teste, yhat, }\DataTypeTok{main=} \StringTok{"Job Satisfaction - Base de teste vs Previsão na base de teste"}\NormalTok{)}
\end{Highlighting}
\end{Shaded}

\includegraphics{Trabalho_files/figure-latex/unnamed-chunk-20-1.pdf}

\begin{Shaded}
\begin{Highlighting}[]
\CommentTok{# Model performance metrics}
\KeywordTok{data.frame}\NormalTok{(}
  \DataTypeTok{RMSE =} \KeywordTok{RMSE}\NormalTok{(yhat, y.teste),}
  \DataTypeTok{Rsquare =} \KeywordTok{R2}\NormalTok{(yhat, y.teste)}
\NormalTok{)}
\end{Highlighting}
\end{Shaded}

\begin{verbatim}
##       RMSE   Rsquare
## 1 1.588161 0.4079989
\end{verbatim}

\subsection{Modelo RIDGE (mesmo que o lasso mas com alpha =
0)}\label{modelo-ridge-mesmo-que-o-lasso-mas-com-alpha-0}

\subsubsection{Escolha do grau de regularização
(lambda)}\label{escolha-do-grau-de-regularizacao-lambda}

\begin{Shaded}
\begin{Highlighting}[]
\CommentTok{# Escolha do grau de regularização (lambda)}
\NormalTok{cv.out <-}\StringTok{ }\KeywordTok{cv.glmnet}\NormalTok{(x.treino, y.treino, }\DataTypeTok{alpha=}\DecValTok{0}\NormalTok{)}
\KeywordTok{plot}\NormalTok{(cv.out)}
\end{Highlighting}
\end{Shaded}

\includegraphics{Trabalho_files/figure-latex/unnamed-chunk-22-1.pdf}

\subsubsection{Analisando os coeficientes deste modelo e erro no
conjunto de
validação}\label{analisando-os-coeficientes-deste-modelo-e-erro-no-conjunto-de-validacao}

\begin{Shaded}
\begin{Highlighting}[]
\CommentTok{# Modelo}
\NormalTok{lasso.mod=}\KeywordTok{glmnet}\NormalTok{(}\DataTypeTok{x=}\NormalTok{x.treino, }\DataTypeTok{y=}\NormalTok{y.treino, }\DataTypeTok{alpha=}\DecValTok{0}\NormalTok{, }\DataTypeTok{lambda=}\NormalTok{cv.out}\OperatorTok{$}\NormalTok{lambda.min)}

\CommentTok{# vamos olhar os coeficientes deste modelo e erro no conjunto de validação}
\NormalTok{resultPredict <-}\StringTok{ }\KeywordTok{predict}\NormalTok{(lasso.mod, }\DataTypeTok{s=}\NormalTok{ cv.out}\OperatorTok{$}\NormalTok{lambda.min , }\DataTypeTok{type=}\StringTok{"coefficients"}\NormalTok{)}
\KeywordTok{head}\NormalTok{(resultPredict)}
\end{Highlighting}
\end{Shaded}

\begin{verbatim}
## 6 x 1 sparse Matrix of class "dgCMatrix"
##                                                                 1
## (Intercept)                                            5.62784232
## ProgramHobbyYes, both                                  0.01553087
## ProgramHobbyYes, I contribute to open source projects  0.04225209
## ProgramHobbyYes, I program as a hobby                 -0.01091579
## CountryArgentina                                       0.02457481
## CountryArmenia                                        -0.15085632
\end{verbatim}

\begin{Shaded}
\begin{Highlighting}[]
\NormalTok{yhat <-}\StringTok{ }\KeywordTok{predict}\NormalTok{(lasso.mod, }\DataTypeTok{s =}\NormalTok{ cv.out}\OperatorTok{$}\NormalTok{lambda.min, }\DataTypeTok{type=}\StringTok{"response"}\NormalTok{, }\DataTypeTok{newx=}\NormalTok{x.teste)}
\KeywordTok{mean}\NormalTok{((yhat}\OperatorTok{-}\NormalTok{y.teste)}\OperatorTok{^}\DecValTok{2}\NormalTok{)}
\end{Highlighting}
\end{Shaded}

\begin{verbatim}
## [1] 3.554289
\end{verbatim}

\subsubsection{Job Satisfaction - Base de
treino}\label{job-satisfaction---base-de-treino-1}

\begin{Shaded}
\begin{Highlighting}[]
\KeywordTok{par}\NormalTok{(}\DataTypeTok{mfrow=}\KeywordTok{c}\NormalTok{(}\DecValTok{2}\NormalTok{,}\DecValTok{2}\NormalTok{))}
\KeywordTok{hist}\NormalTok{(y.treino, }\DataTypeTok{main=} \StringTok{"Job Satisfaction - Base de treino"}\NormalTok{, }\DataTypeTok{col=}\StringTok{"lightblue"}\NormalTok{)}
\end{Highlighting}
\end{Shaded}

\includegraphics{Trabalho_files/figure-latex/unnamed-chunk-25-1.pdf}

\subsubsection{Job Satisfaction - Base de
teste}\label{job-satisfaction---base-de-teste-1}

\begin{Shaded}
\begin{Highlighting}[]
\KeywordTok{hist}\NormalTok{(y.teste, }\DataTypeTok{main=} \StringTok{"Job Satisfaction - Base de teste"}\NormalTok{, }\DataTypeTok{col=}\StringTok{"lightblue"}\NormalTok{)}
\end{Highlighting}
\end{Shaded}

\includegraphics{Trabalho_files/figure-latex/unnamed-chunk-26-1.pdf}

\subsubsection{Job Satisfaction - Previsão na base de
teste}\label{job-satisfaction---previsao-na-base-de-teste-1}

\begin{Shaded}
\begin{Highlighting}[]
\KeywordTok{hist}\NormalTok{(yhat, }\DataTypeTok{main=} \StringTok{"Job Satisfaction - Previsão na base de teste"}\NormalTok{, }\DataTypeTok{col=}\StringTok{"lightblue"}\NormalTok{)}
\end{Highlighting}
\end{Shaded}

\includegraphics{Trabalho_files/figure-latex/unnamed-chunk-27-1.pdf}

\subsubsection{Job Satisfaction - Base de teste vs Previsão na base de
teste}\label{job-satisfaction---base-de-teste-vs-previsao-na-base-de-teste}

\begin{Shaded}
\begin{Highlighting}[]
\KeywordTok{plot}\NormalTok{(y.teste, yhat, }\DataTypeTok{main=} \StringTok{"Job Satisfaction - Base de teste vs Previsão na base de teste"}\NormalTok{)}
\end{Highlighting}
\end{Shaded}

\includegraphics{Trabalho_files/figure-latex/unnamed-chunk-28-1.pdf}

\subsubsection{Métricas de desempenho do
modelo}\label{metricas-de-desempenho-do-modelo-1}

\begin{Shaded}
\begin{Highlighting}[]
\CommentTok{# Model performance metrics}
\KeywordTok{data.frame}\NormalTok{(}
  \DataTypeTok{RMSE =} \KeywordTok{RMSE}\NormalTok{(yhat, y.teste),}
  \DataTypeTok{Rsquare =} \KeywordTok{R2}\NormalTok{(yhat, y.teste)}
\NormalTok{)}
\end{Highlighting}
\end{Shaded}

\begin{verbatim}
##       RMSE   Rsquare
## 1 1.885282 0.2001939
\end{verbatim}


\end{document}
